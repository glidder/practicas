\documentclass[a4paper]{article}
%%%%%Paquetes%%%%%
\usepackage[T1]{fontenc}
\usepackage[utf8]{inputenc}
\usepackage[spanish]{babel}
\usepackage{amsmath,amssymb,eucal,mathrsfs}
\usepackage[svgnames,x11names]{xcolor}
\usepackage{colortbl}
\usepackage{Estilos/MiEstilo}
%\usepackage{Estilos/Informe}
\usepackage{amsmath}
\usepackage{times}
\usepackage{color}
\usepackage{listings}
\usepackage{graphicx}
\usepackage{caption}
\usepackage{float}
\usepackage{algpseudocode}
\usepackage{algorithm}
	\setcounter{tocdepth}{2} %Mostrar solo 3 niveles en el índice
		\makeindex	%índice de palabras
		\title{Computación Cuantica \\ Modelos de Computacion}
		\author{Luis José Quintana Bolaño}
		\date{Enero, 2014}

\begin{document}
	\maketitle
	\begin{abstract}
	    Este documento comprende un relativamente breve análisis del paradigma de la computación cuantica. Incluyendo como mínimo definición, ejemplos de cálculo, definición de función computable, relaciones con otros modelos de computación conocidos.
  	\end{abstract}
  	
  	\section{Definición y orígenes}
  		La computación cuantica es un paradigma de computación basado en "qubits", en lugar de los clásicos bits. Esta diferencia fundamental da lugar nuevas puertas lógicas, que permite la creación de nuevos algoritmos.
  		http://en.wikipedia.org/wiki/Quantum_computer
  		\section{Modelo de computación}
  		http://en.wikipedia.org/wiki/Quantum_Turing_machine
  	\appendix
	\section{Descripción del conjunto de Mandelbrot}
		
	\subsection{Pseudocódigo} 
		\begin{algorithm}
			\caption{Algoritmo "tiempo de escape"}
			\begin{algorithmic}[1]
			\For{cada pixel $(Px, Py)$ de la pantalla}
				\State $x0 \gets$ coordenada $x$ del pixel a escala Mandelbrot
  				\State $y0 \gets$ coordenada $y$ del pixel a escala Mandelbrot
  				\State $x \gets 0.0$
  				\State $y \gets 0.0$
  				\State $iteracion \gets 0$
				\State $iteracion\_maxima \gets 1000$
				\While{$(x*x + y*y < 2*2 ) \land (iteracion < iteracion\_maxima)$}
					\State $auxiliar \gets x*x - y*y + x0$
    				\State $y \gets 2*x*y + y0$
    				\State $x \gets auxiliar$
    				\State $iteracion \gets iteracion + 1$
				\EndWhile
				\State $color \gets paleta[iteration]$
  				\State $pintar(Px, Py, color)$
			\EndFor
			\end{algorithmic}
		\end{algorithm}			
		%\lstinputlisting[language=C]{scapetime}
	\\Nótese que el algoritmo no utiliza numeros complejos, pero los simula mediante dos números reales.
	\newpage
	\subsection{Dimensión Fractal}
\end{document}
