\documentclass[a4paper]{article}
%%%%%Paquetes%%%%%
\usepackage[T1]{fontenc}
\usepackage[utf8]{inputenc}
\usepackage[spanish]{babel}
\usepackage{amsmath,amssymb,eucal,mathrsfs}
\usepackage[svgnames,x11names]{xcolor}
\usepackage{colortbl}
\usepackage{Estilos/MiEstilo}
%\usepackage{Estilos/Informe}
\usepackage{times}
\usepackage{color}
\usepackage{listings}
\usepackage{graphicx}
\usepackage{caption}

	\setcounter{tocdepth}{2} %Mostrar solo 3 niveles en el índice
		\makeindex	%índice de palabras
		\title{Modelos de Computación.\\ Práctica 1. }
		\author{Luis José Quintana Bolaño}
		\date{\today}

\begin{document}
		\maketitle
		\begin{abstract}
		    Especificación de las funciones determinadas para cada ejemplo de L tras una batería de pruebas.
  		\end{abstract}

  		\section{Ejemplo ej1}
  		    \subsection{Batería de pruebas}
			\begin{center}
  		        \begin{tabular}{|c|c|c|} \hline
  		            \multicolumn{2}{|l|}{Entrada}&\multicolumn{1}{l|}{Salida}\\
			    X1 & X2 & Y \\
  		            \hline
  		        0 & 0 & 0 \\
			    1 & 0 & 1 \\
			    0 & 1 & 1 \\
			    1 & 1 & 2 \\
			    2 & 3 & 5 \\
  		            \hline
  		        \end{tabular} \\
			\end{center}
  		    \subsection{Función}
  		        Se corresponde con la función:
  		        $$f: \mathbb{N}^2 \longrightarrow \mathbb{N}$$
  		        \begin{equation*}
  		        	f(x_1,x_2) = x_1 + x_2 
  		        \end{equation*}

  		\section{Ejemplo ej2}
  			\subsection{Batería de pruebas}
  			\begin{center}
  				\begin{tabular}{|c|c|} \hline
  					\multicolumn{1}{|l|}{Entrada}&\multicolumn{1}{l|}{Salida}\\
  				X & Y \\
  					\hline
  				0 & 0 \\
  				1 & 3 \\
  				2 & 6 \\
  					\hline
  				\end{tabular} \\
  			\end{center}
  			\subsection{Función}
  				Se corresponde con la función:
  				$$f: \mathbb{N} \longrightarrow \mathbb{N}$$
  				\begin{equation*}
  					f(x) = x * 3
  				\end{equation*}

  		\section{Ejemplo ej5}
  			\subsection{Batería de pruebas}
  			\begin{center}
  				\begin{tabular}{|c|c|} \hline
  					\multicolumn{1}{|l|}{Entrada}&\multicolumn{1}{l|}{Salida}\\
  				X & Y \\
  					\hline
  				0 & 1 \\
  				1 & 0 \\
  				2 & 1 \\
  				5 & 0 \\
  					\hline
  				\end{tabular} \\
  			\end{center}
  			\subsection{Función}
  				Se corresponde con la función:
  				$$f: \mathbb{N} \longrightarrow \{0, 1\}$$
  				\begin{equation*}
  					f(x) = \left\{ 
  					\begin{array}{rcl}
  						1 & \text{si } x=2k, & k \in \mathbb{N}, \\
  						0 & \text{si } x=2k+1, & k \in \mathbb{N}
  					\end{array} \right.
  				\end{equation*}

  		\section{Ejemplo ej6}
  			\subsection{Batería de pruebas}
  			\begin{center}
  				\begin{tabular}{|c|c|} \hline
  					\multicolumn{1}{|l|}{Entrada}&\multicolumn{1}{l|}{Salida}\\
  				X & Y \\
  					\hline
  				0 & 1 \\
  				1 & $\uparrow$ \\
  				2 & 1 \\
  				3 & $\uparrow$ \\
  				4 & 1 \\
  					\hline
  				\end{tabular} \\
  			\end{center}
  			\subsection{Función}
  				Se corresponde con la función:
  				$$f:\subseteq \mathbb{N} \longrightarrow \{1\}$$
  				\begin{equation*}
  					f(x) = \left\{
  					\begin{array}{rcl}
  						1 & \text{si } x=2k, & k \in \mathbb{N}, \\
  						\uparrow & \text{si } x=2k+1, & K \in \mathbb{N}
  					\end{array} \right.
  				\end{equation*}
  				La función es parcialmente computable, estando solo definida para los números pares.

  		\section{Ejemplo ej7}
  			\subsection{Batería de pruebas}
  			\begin{center}
  				\begin{tabular}{|c|c|c|} \hline
  					\multicolumn{2}{|l|}{Entrada}&\multicolumn{1}{l|}{Salida}\\
  				X1 & X2 & Y \\
  					\hline
  				0 & 0 & 1 \\
  				0 & 1 & 0 \\
  				1 & 0 & 0 \\
  				1 & 1 & 1 \\
  				2 & 2 & 1 \\
  				3 & 5 & 0 \\
  					\hline
  				\end{tabular}\\
  			\end{center}
  			\subsection{Función}
  				Se corresponde con la función:
  				$$f:\mathbb{N}^2 \longrightarrow \{0, 1\}$$
  				\begin{equation*}
  					f(x_1, x_2) = \left\{
  					\begin{array}{rl}
  					0 & \text{si } x_1 \not =x_2, \\
  					1 & \text{si } x_1= x_2 \\
  					\end{array} \right.
  				\end{equation*}

  		\section{Ejemplo ej8}
  			\subsection{Batería de pruebas}
  			\begin{center}
  				\begin{tabular}{|c|c|} \hline
  					\multicolumn{1}{|l|}{Entrada}&\multicolumn{1}{l|}{Salida}\\
  				X & Y \\
  					\hline
  				0 & 0 \\
  				1 & 0 \\
  				2 & 0 \\
  				4 & 2 \\
  				9 & 3 \\
  				10 & 3 \\
  				16 & 4 \\
  					\hline
  				\end{tabular}\\
  			\end{center}
  			\subsection{Función}
  				Se corresponde con la función:
  				$$f:\mathbb{N} \longrightarrow \mathbb{N}$$
  				\begin{equation*}
  					f(x) = \lfloor \sqrt{x} \rfloor
  				\end{equation*}

  		\section{Ejemplo ej9}
  			\subsection{Batería de pruebas}
  			\begin{center}
  				\begin{tabular}{|c|c|c|} \hline
  					\multicolumn{2}{|l|}{Entrada}&\multicolumn{1}{l|}{Salida}\\
  				X1 & X2 & Y \\
  					\hline
  				0 & 0 & 0 \\
  				1 & 0 & 0 \\
  				0 & 1 & 0 \\
  				1 & 1 & 1 \\
  				2 & 2 & 2 \\
  				5 & 5 & 5 \\
  				5 & 2 & 1 \\
  				4 & 8 & 4 \\
  					\hline
  				\end{tabular}\\
  			\end{center}
  			\subsection{Función}
  				Se corresponde con la función:
  				$$f:\mathbb{N}^2 \longrightarrow \mathbb{N}$$
  				\begin{equation*}
  					f(x_1, x_2) = \mathrm{mcd}(x_1,x_2)
  				\end{equation*}

  		 \section{Ejemplo ej10}
  		 	\subsection{Batería de pruebas}
  		 	\begin{center}
  		 		\begin{tabular}{|c|c|c|} \hline
  		 			\multicolumn{2}{|l|}{Entrada}&\multicolumn{1}{l|}{Salida}\\
  		 		X1 & X2 & Y \\
  		 			\hline
  		 		0 & 0 & 0 \\
  		 		1 & 0 & 2 \\
  		 		0 & 1 & 3 \\
  		 		1 & 1 & 0 \\
  		 		5 & 2 & 6 \\
  		 		2 & 5 & 7 \\
  		 			\hline
  		 		\end{tabular}\\
  		 	\end{center}
  		 	\subsection{Función}
  		 		Se corresponde con la función:
  		 		$$f:\mathbb{N}^2 \longrightarrow \mathbb{N}$$
  		 		\begin{equation*}
  		 			f(x_1, x_2) = \left\{
  		 			\begin{array}{rl}
  					2(x_1 - x_2) & \text{si } x_1 > x_2, \\
  					2(x_2 - x_1)+1 & \text{si } x_1 < x_2 \\
  					\end{array} \right.
  				\end{equation*}

  		\section{Ejemplo ej14}
  			\subsection{Batería de pruebas}
  			\begin{center}
  				\begin{tabular}{|c|c|} \hline
  					\multicolumn{1}{|l|}{Entrada}&\multicolumn{1}{l|}{Salida}\\
  				X & Y \\
  					\hline
  				0 & $\uparrow$ \\
  				1 & $\uparrow$ \\
  					\hline
  				\end{tabular}\\
  			\end{center}
  			\subsection{Función}
  				No se corresponde con ninguna función.

  		\section{Ejemplo ej15}
  			\subsection{Batería de pruebas}
  			\begin{center}
  				\begin{tabular}{|c|c|} \hline
  					\multicolumn{1}{|l|}{Entrada}&\multicolumn{1}{l|}{Salida}\\
  				X & Y \\
  					\hline
  				0 & $\uparrow$ \\
  				1 & 0 \\
  				4 & 0 \\
  					\hline
  				\end{tabular}\\
  			\end{center}
  			\subsection{Función}
  				Se corresponde con la función:
  				$$f:\subseteq \mathbb{N} \longrightarrow \{0\}$$
  				\begin{equation*}
  					f(x) = \left\{
  					\begin{array}{rl}
  						0 & \text{si } x>0, \\
  						\uparrow & \text{si } x=0 \\
  					\end{array} \right.
  				\end{equation*}
  				La función es parcialmente computable, estando definida para $\mathbb{N}-\{0\}$.

  		\section{Ejemplo ej17}
  			\subsection{Batería de pruebas}
  			\begin{center}
  				\begin{tabular}{|c|c|c|} \hline
  					\multicolumn{2}{|l|}{Entrada}&\multicolumn{1}{l|}{Salida}\\
  				X1 & X2 & Y \\
  					\hline
  				0 & 0 & $\uparrow$ \\
  				1 & 0 & $\uparrow$ \\
  				0 & 1 & 0 \\
  				1 & 1 & 1 \\
  				2 & 1 & 2 \\
  				5 & 1 & 5 \\
  					\hline
  				\end{tabular}\\
  			\end{center}
  			\subsection{Función}
  				Se corresponde con la función:
  				$$f:\subseteq \mathbb{N}^2 \longrightarrow \{0\}$$
  				\begin{equation*}
  					f(x_1, x_2) = \left\{
  					\begin{array}{rl}
  						x_1 & \text{si } x_2 > 0, \\
  						\uparrow & \text{si } x_2=0 \\
  					\end{array} \right.
  				\end{equation*}
  				La función es parcialmente computable, estando definida para $x_1 \in \mathbb{N} \text{y } x_2 \in \mathbb{N}-\{0\}$.











\end{document}
