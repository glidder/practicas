\documentclass[a4paper]{article}
%%%%%Paquetes%%%%%
\usepackage[T1]{fontenc}
\usepackage[utf8]{inputenc}
\usepackage[spanish]{babel}
\usepackage{amsmath,amssymb,eucal,mathrsfs}
\usepackage[svgnames,x11names]{xcolor}
\usepackage{colortbl}
\usepackage{Estilos/MiEstilo}
%\usepackage{Estilos/Informe}
\usepackage{amsmath}
\usepackage{times}
\usepackage{color}
\usepackage{listings}
\usepackage{graphicx}
\usepackage{caption}
\usepackage{float}
	\setcounter{tocdepth}{2} %Mostrar solo 3 niveles en el índice
		\makeindex	%índice de palabras
		\title{Modelos de Computación.\\ Práctica 8. }
		\author{Luis José Quintana Bolaño}
		\date{\today}

\begin{document}
	\maketitle
	\begin{abstract}
	    Breve comentario sobre la computabilidad y el conjunto de mandelbrot.
  	\end{abstract}
	\section{}
		Hasta ahora no se conoce la respuesta a si el conjunto de Mandelbrot es computable en modelos de computación con números reales basados en el analisis computacional, que se corresponden más estrechamente a la noción intuitiva de "trazarlo en un ordenador". Hertling demostró que el conunto de Mandelbrot es computable en este modelo si la conjetura de hiperbolicidad es cierta. 
		
\end{document}
