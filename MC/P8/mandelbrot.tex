\documentclass[a4paper]{article}
%%%%%Paquetes%%%%%
\usepackage[T1]{fontenc}
\usepackage[utf8]{inputenc}
\usepackage[spanish]{babel}
\usepackage{amsmath,amssymb,eucal,mathrsfs}
\usepackage[svgnames,x11names]{xcolor}
\usepackage{colortbl}
\usepackage{Estilos/MiEstilo}
%\usepackage{Estilos/Informe}
\usepackage{amsmath}
\usepackage{times}
\usepackage{color}
\usepackage{listings}
\usepackage{graphicx}
\usepackage{caption}
\usepackage{float}
	\setcounter{tocdepth}{2} %Mostrar solo 3 niveles en el índice
		\makeindex	%índice de palabras
		\title{Modelos de Computación.\\ Práctica 8. }
		\author{Luis José Quintana Bolaño}
		\date{\today}

\begin{document}
	\maketitle
	\begin{abstract}
	    Descipción del conjunto de Mandelbrot, un conjunto matemático de puntos cuya frontera conforma una de las más conocidas formas fractales. Está estrechamente relacionado con el conjunto de Julia. El nombre del conunto le fue dado en honor a Benoît Mandelbrot, matemático conocido por su trabajo en el campo de los fractales y principal responsable de su popularidad.
  	\end{abstract}
	\section{}
		
\end{document}
