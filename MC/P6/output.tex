\documentclass[a4paper]{article}
%%%%%Paquetes%%%%%
\usepackage[T1]{fontenc}
\usepackage[utf8]{inputenc}
\usepackage[spanish]{babel}
\usepackage{amsmath,amssymb,eucal,mathrsfs}
\usepackage[svgnames,x11names]{xcolor}
\usepackage{colortbl}
\usepackage{Estilos/MiEstilo}
%\usepackage{Estilos/Informe}
\usepackage{amsmath}
\usepackage{times}
\usepackage{color}
\usepackage{listings}
\usepackage{graphicx}
\usepackage{caption}
\usepackage{float}
	\setcounter{tocdepth}{2} %Mostrar solo 3 niveles en el índice
		\makeindex	%índice de palabras
		\title{Modelos de Computación.\\ Práctica 6. }
		\author{Luis José Quintana Bolaño}
		\date{\today}

\begin{document}
	\maketitle
	\begin{abstract}
	    Breve comentario sobre computación universal en Life.
  	\end{abstract}
	\section{Ejercicio 2}
		Tras la conjetura de Conway sobre la imposibilidad del crecimiento eterno de una comunidad de células en Life, varias estructuras surgieron que probaban falsa dicha asunción (siendo la primera la "pistola de deslizadores de Gospel". La existencia de estas estructuras generadoras y otras estables permite la construcción de puertas lógicas y celdas de memoria en el juego, así como estructuras que actúan como autómatas de estados finitos. Es decir, que tiene el mismo poder computacional que una máquina universal de Turing, lo que otorga al juego la capacidad teórica de albergar tanta potencia como una computadora con memoria y tiempo ilimitadas, es decir, que es Turing-completo.
\end{document}
